\documentclass{beamer}
\usetheme{Warsaw}
\usepackage{graphicx}
\usepackage{listings}
\usepackage[utf8]{inputenc} 
\usepackage[polish]{babel}  
\usepackage{verbatim}
\usepackage[T1]{fontenc}
\usepackage[utf8]{inputenc}
\usepackage[polish]{babel}


\title{Zastosowanie metod formalnych do weryfikacji struktur wskaźnikowych w systemie PVS}
\author{Karol Kozlowski,}
\institute{Wydzial elektryczny, Politechnika Warszawska}
\date{\today}

\begin{document}

\begin{frame}
\titlepage
\end{frame}

\begin{frame}
\tableofcontents
\end{frame}

\section{Wprowadzenie do metod formalnych}
\begin{frame}{Metody formalne w inżynierii oprogramowania}
\begin{block}{Dlaczego metody formalne?}
\begin{itemize}
\item Krytyczne systemy: medyczne, kosmiczne, transportowe
\item Koszt błędów: katastrofy vs koszt wdrożenia
\item Przykład: NASA i system PVS
\end{itemize}
\end{block}

\begin{exampleblock}{Kluczowe zalety}
\begin{itemize}
\item Pełna weryfikacja własności
\item Wykrywanie wycieków pamięci
\item Gwarancja niezmienników strukturalnych
\end{itemize}
\end{exampleblock}
\end{frame}

\section{Wprowadzenie do PVS}
\begin{frame}[fragile]{System PVS w pigułce}


\begin{itemize}
\item Logika wyższego rzędu
\item Mechanizm dowodzenia twierdzeń
\item Automatyczne generowanie TCC (Type Correctness Conditions)
\item Parametryzacja teorii
\end{itemize}


\begin{verbatim}
  pointer_env [P: TYPE, T: TYPE]: THEORY
  BEGIN
    pointer: TYPE = P + {nil}
    env: TYPE = [pointer -> (T + {undefined})]
  END pointer_env
\end{verbatim}
\end{frame}

\section{Wyzwania dla struktur wskaźnikowych}
\begin{frame}{Problem: Weryfikacja struktur dynamicznych}
\begin{alertblock}{Główne wyzwania}
\begin{itemize}
\item Dynamiczna alokacja pamięci
\item Aliasing wskaźników
\item Zachowanie niezmienników po operacjach
\item Cykliczne struktury danych
\end{itemize}
\end{alertblock}

\begin{exampleblock}{Przykładowa specyfikacja listy}
\begin{itemize}
\item $\forall l_1 \neq l_2 \Rightarrow \neg \exists n \in (l_1 \cap l_2)$
\item $\forall p \in \text{pointer} \Rightarrow \exists! l: p \in l$
\end{itemize}
\end{exampleblock}
\end{frame}

\section{Metodologia pracy}
\begin{frame}[fragile]{Proces weryfikacji w PVS}
\begin{enumerate}
\item Modelowanie środowiska wskaźnikowego
\item Definicja niezmienników strukturalnych
\item Generowanie i dowodzenie TCC
\item Specyfikacja operacji (predykaty)
\item Dowód zachowania niezmienników
\end{enumerate}

\begin{verbatim}

list_member?(l: pointer, vl: list): bool = 
  IF nil?(vl) THEN false
  ELSE l = vl OR list_member?(l, next(vl))
  ENDIF
\end{verbatim}
\end{frame}

\section{Studium przypadku: lista}
\begin{frame}[fragile]{Studium przypadku: Lista jednokierunkowa}
\begin{block}{Kluczowe niezmienniki}
\begin{itemize}
\item Spójność typów (TCC)
\item Rozłączność list
\item Pełna pokrycie pamięci
\item Brak cykli
\end{itemize}
\end{block}

\begin{exampleblock}{Przykładowe twierdzenie}
\begin{verbatim}
  member_last: LEMMA 
  FORALL (vl: list): 
    NOT nil?(vl) => list_member?(last(vl), vl)
\end{verbatim}
\end{exampleblock}
\end{frame}

\section{Wyzwania i ograniczenia}
\begin{frame}{Wyzwania i wnioski}
\begin{alertblock}{Główne trudności}
\begin{itemize}
\item Czasochłonność dowodów (do 1 tygodnia na predykat)
\item Ograniczenia PVS w pracy z wieloma teoriami
\item Trudności w automatyzacji dla grafów
\end{itemize}
\end{alertblock}

\begin{block}{Podsumowanie}
\begin{itemize}
\item Metoda skuteczna dla list i drzew
\item Wymaga dużego nakładu pracy
\item Obiecujące wyniki dla systemów krytycznych
\end{itemize}
\end{block}
\end{frame}

\section{Zastosowanie metod formalnych: TLA+}
\begin{frame}{Czym jest TLA+?}
  \begin{itemize}
      \item TLA+ (Temporal Logic of Actions) - język specyfikacji formalnej opracowany przez Lesliego Lamporta.
      \item Służy do opisu systemów współbieżnych i rozproszonych.
      \item System definiowany jako zbiór zmiennych i akcji - przejść między stanami.
      \item Deklaratywny - skupia się na zachowaniu, nie na implementacji.
      \item Bazuje na logice temporalnej i matematyce zbiorów.
  \end{itemize}
\end{frame}

\begin{frame}{Narzędzie TLC i model checking}
  \begin{itemize}
      \item Weryfikacja poprawności specyfikacji odbywa się z użyciem narzędzia TLC.
      \item TLC używa techniki model checking:
      \begin{itemize}
          \item automatyczne przeszukiwanie przestrzeni stanów,
          \item sprawdzanie właściwości bezpieczeństwa i ciągłości.
      \end{itemize}
  \end{itemize}
\end{frame}

\begin{frame}{Metoda 1: Modelowanie matematyczne}
  \begin{itemize}
      \item Określenie przestrzeni stanów i przejść między nimi (akcji).
      \item Przykład: bufor FIFO - kolejka, enqueue, dequeue.
      \item Modelowanie oparte na funkcjach, zbiorach, wektorach.
      \item Oddzielenie specyfikacji od implementacji.
  \end{itemize}
\end{frame}

\begin{frame}{Metoda 2: Inwarianty bezpieczeństwa}
  \begin{itemize}
      \item Inwarianty - warunki logiczne obowiązujące w każdym stanie.
      \item Przykład: \texttt{Len(queue) >= 0}.
      \item Weryfikacja wszystkich możliwych trajektorii przez TLC.
      \item Zapobieganie błędom - np. usunięcie z pustego bufora.
  \end{itemize}
\end{frame}

\begin{frame}{Metoda 3: Własności temporalne}
  \begin{itemize}
      \item Logika temporalna LTL - analiza zachowań w czasie.
      \item Przykłady:
      \begin{itemize}
          \item „zawsze po A następuje B”,
          \item „stan X zostanie osiągnięty w przyszłości”.
      \end{itemize}
      \item Formuły: bezpieczeństwo (\texttt{[] (LockCount <= 1)}), ciągłość (\texttt{<>(queue = << >>)}).
  \end{itemize}
\end{frame}

\begin{frame}{Zastosowania TLA+}
  \begin{itemize}
      \item Firmy: Amazon, Microsoft, Google.
      \item Projektowanie i weryfikacja systemów:
      \begin{itemize}
          \item rozproszonych,
          \item bazodanowych,
          \item komunikacyjnych,
          \item chmurowych.
      \end{itemize}
      \item Identyfikacja błędów przed implementacją - kluczowa dla systemów krytycznych.
  \end{itemize}
\end{frame}


\begin{frame}{Bibliografia}
\begin{thebibliography}{9}
  \bibitem{lamport2002specifying}
  Leslie Lamport. 
  \emph{Specifying Systems: The TLA+ Language and Tools for Hardware and Software Engineers}. 

  \bibitem{newcombe2015amazon}
  Chris Newcombe et al. 
  \emph{How Amazon Web Services Uses Formal Methods}. 
  
  \bibitem{konnov2019tla}
  Igor Konnov, Jure Kukovec, Thanh-Hai Tran. 
  \emph{TLA+ Model Checking Made Symbolic}. 
  
  \bibitem{wayne2018practical}
  Hillel Wayne. 
  \emph{Practical TLA+: Planning Driven Development}. 
  Lospinato Books, 2018. 
  
  \bibitem{pvs} 
  S. Owre et al. 
  \emph{PVS System Guide}. 
  SRI International, 1999.
  
  \bibitem{paper} 
  S. Poreda. 
  \emph{Wykorzystanie metod formalnych do specyfikacji struktur wskaźnikowych}. 
  Uniwersytet Warszawski, 2023.
\end{thebibliography}
\end{frame}
\end{document}