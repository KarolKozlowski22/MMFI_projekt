\documentclass{beamer}
\usetheme{Warsaw}
\usepackage{graphicx}
\usepackage{listings}
\usepackage[utf8]{inputenc} 
\usepackage[polish]{babel}  
\usepackage{verbatim}

\title{Zastosowanie metod formalnych do weryfikacji struktur wskaźnikowych w systemie PVS}
\author{Karol Kozlowski,}
\institute{Wydzial elektryczny, Politechnika Warszawska}
\date{\today}

\begin{document}

\begin{frame}
\titlepage
\end{frame}

\begin{frame}
\tableofcontents
\end{frame}

\section{Wprowadzenie do metod formalnych}
\begin{frame}{Metody formalne w inżynierii oprogramowania}
\begin{block}{Dlaczego metody formalne?}
\begin{itemize}
\item Krytyczne systemy: medyczne, kosmiczne, transportowe
\item Koszt błędów: katastrofy vs koszt wdrożenia
\item Przykład: NASA i system PVS
\end{itemize}
\end{block}

\begin{exampleblock}{Kluczowe zalety}
\begin{itemize}
\item Pełna weryfikacja własności
\item Wykrywanie wycieków pamięci
\item Gwarancja niezmienników strukturalnych
\end{itemize}
\end{exampleblock}
\end{frame}

\section{Wprowadzenie do PVS}
\begin{frame}[fragile]{System PVS w pigułce}


\begin{itemize}
\item Logika wyższego rzędu
\item Mechanizm dowodzenia twierdzeń
\item Automatyczne generowanie TCC (Type Correctness Conditions)
\item Parametryzacja teorii
\end{itemize}


\begin{verbatim}
  pointer_env [P: TYPE, T: TYPE]: THEORY
  BEGIN
    pointer: TYPE = P + {nil}
    env: TYPE = [pointer -> (T + {undefined})]
  END pointer_env
\end{verbatim}
\end{frame}

\section{Wyzwania dla struktur wskaźnikowych}
\begin{frame}{Problem: Weryfikacja struktur dynamicznych}
\begin{alertblock}{Główne wyzwania}
\begin{itemize}
\item Dynamiczna alokacja pamięci
\item Aliasing wskaźników
\item Zachowanie niezmienników po operacjach
\item Cykliczne struktury danych
\end{itemize}
\end{alertblock}

\begin{exampleblock}{Przykładowa specyfikacja listy}
\begin{itemize}
\item $\forall l_1 \neq l_2 \Rightarrow \neg \exists n \in (l_1 \cap l_2)$
\item $\forall p \in \text{pointer} \Rightarrow \exists! l: p \in l$
\end{itemize}
\end{exampleblock}
\end{frame}

\section{Metodologia pracy}
\begin{frame}[fragile]{Proces weryfikacji w PVS}
\begin{enumerate}
\item Modelowanie środowiska wskaźnikowego
\item Definicja niezmienników strukturalnych
\item Generowanie i dowodzenie TCC
\item Specyfikacja operacji (predykaty)
\item Dowód zachowania niezmienników
\end{enumerate}

\begin{verbatim}

list_member?(l: pointer, vl: list): bool = 
  IF nil?(vl) THEN false
  ELSE l = vl OR list_member?(l, next(vl))
  ENDIF
\end{verbatim}
\end{frame}

\section{Studium przypadku: lista}
\begin{frame}[fragile]{Studium przypadku: Lista jednokierunkowa}
\begin{block}{Kluczowe niezmienniki}
\begin{itemize}
\item Spójność typów (TCC)
\item Rozłączność list
\item Pełna pokrycie pamięci
\item Brak cykli
\end{itemize}
\end{block}

\begin{exampleblock}{Przykładowe twierdzenie}
\begin{verbatim}
  member_last: LEMMA 
  FORALL (vl: list): 
    NOT nil?(vl) => list_member?(last(vl), vl)
\end{verbatim}
\end{exampleblock}
\end{frame}

\section{Wyzwania i ograniczenia}
\begin{frame}{Wyzwania i wnioski}
\begin{alertblock}{Główne trudności}
\begin{itemize}
\item Czasochłonność dowodów (do 1 tygodnia na predykat)
\item Ograniczenia PVS w pracy z wieloma teoriami
\item Trudności w automatyzacji dla grafów
\end{itemize}
\end{alertblock}

\begin{block}{Podsumowanie}
\begin{itemize}
\item Metoda skuteczna dla list i drzew
\item Wymaga dużego nakładu pracy
\item Obiecujące wyniki dla systemów krytycznych
\end{itemize}
\end{block}
\end{frame}

\begin{frame}{Bibliografia}
\begin{thebibliography}{9}
\bibitem{pvs} 
S. Owre et al. 
\emph{PVS System Guide}. 
SRI International, 1999.

\bibitem{paper} 
S. Poreda. 
\emph{Wykorzystanie metod formalnych do specyfikacji struktur wskaźnikowych}. 
Uniwersytet Warszawski, 2023.
\end{thebibliography}
\end{frame}

\end{document}