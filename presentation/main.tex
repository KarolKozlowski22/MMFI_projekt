\documentclass{beamer}
\usetheme{Warsaw}
\usepackage{graphicx}
\usepackage{listings}
\usepackage{verbatim}

\usepackage[utf8]{inputenc} 
\usepackage[polish]{babel}  
\usepackage[T1]{fontenc}
\usepackage{verbatim}
\usepackage{amsmath, amssymb}
\usepackage{tcolorbox}
\usepackage{tabularx}

\title{Zastosowanie metod formalnych}
\author{Karol Kozlowski, Katarzyna Mielęcka, Jan Gawroński, Piotr Głowacki}
\institute{Wydzial elektryczny, Politechnika Warszawska}
\date{\today}

\begin{document}

\begin{frame}{}
\titlepage
\end{frame}

\begin{frame}{Spis treści}
\tableofcontents
\end{frame}

\section{Wprowadzenie do metod formalnych}
\begin{frame}{Czym są metody formalne?}
\begin{block}{Definicja}
  Metody formalne to techniki matematyczne służące do:
\begin{itemize}
\item Specyfikacji - precyzyjnego opisu systemów za pomocą języków matematycznych (np. logika temporalna, rachunek procesów)
\item Weryfikacji - dowodzenia poprawności systemów poprzez np. model checking (np. narzędzie SPIN) lub dowody twierdzeń (np. Coq, Isabelle)
\item Automatyzacji analizy  wykrywania sprzeczności lub luk w projektach na etapie modelowania (np. weryfikacja protokołów kryptograficznych)
\end{itemize}
\end{block}
\end{frame}

\begin{frame}
\begin{exampleblock}{Kluczowe zalety}
\begin{itemize}
\item Formalne gwarancje poprawności - zapewnienie matematycznie udowodnionej poprawności systemów, szczególnie w przypadku wymagań bezpieczeństwa.
\item Precyzyjna specyfikacja wymagań - eliminacja niejednoznaczności dzięki matematycznym modelom i notacjom.
\item Wykrywanie złożonych błędów - identyfikacja problemów takich jak zakleszczenia (deadlocks) czy warunki wyścigu (race conditions), które trudno wykryć tradycyjnymi metodami testowania.
\end{itemize}
\end{exampleblock}

\end{frame}

\section{Wykorzystanie metod formalnych w systemie PVS}
\subsection{Wprowadzenie do systemu PVS}
\begin{frame}[fragile]{System PVS w pigułce}
  System PVS służy do opracowania i weryfikacji specyfikacji opisujących
  różne zagadnienia. PVS posiada rozbudowaną składnię i umożliwia operowanie w
  logice wyższego rzędu, definiowanie własnych typów i podtypów danych oraz
  tworzenie teorii parametryzowanych \cite{paper}.

\begin{verbatim}
  pointer_env [P: TYPE, T: TYPE]: THEORY
  BEGIN
    pointer: TYPE = P + {nil}
    env: TYPE = [pointer -> (T + {undefined})]
  END pointer_env
\end{verbatim}
\end{frame}

\subsection{Jaki problem rozwiazuje system PVS?}
\begin{frame}{Problem: Weryfikacja struktur dynamicznych}
\begin{alertblock}{Główne wyzwania}
  \begin{itemize}
  \item \textbf{Złożoność struktur wskaźnikowych}:
  \begin{itemize}
      \item Wycieki pamięci
      \item Nieprawidłowe dereferencje
      \item Zakleszczenia w systemach współbieżnych
  \end{itemize}
  
  \item \textbf{Niejednoznaczność specyfikacji}:
  \begin{itemize}
      \item Niewystarczalność logiki pierwszego rzędu
      \item Potrzeba logiki wyższego rzędu (PVS)
  \end{itemize}
  
  \item \textbf{Krytyczne zastosowania}:
  \begin{itemize}
      \item Systemy sterowania (metro, koleje)
      \item Aplikacje medyczne
      \item Systemy awioniki i kosmiczne
  \end{itemize}
\end{itemize}
\end{alertblock}
\end{frame}

\subsection{Rozwiazanie PVS}
\begin{frame}{Rozwiazanie PVS}

    \begin{itemize}
      \item Formalna specyfikacja niezmienników
      \item Automatyczna weryfikacja warunków spójności
      \item Półautomatyczne dowodzenie poprawności
    \end{itemize}
\end{frame}

\subsection{Formalna specyfikacja niezmiennikow}
\begin{frame}[fragile]{Przykład}
  \begin{exampleblock}{Przykład}
    
  \begin{verbatim}
    accessed_disjoint?(accessed? : pred[pointer[P]]) : 
    boolean =
    FORALL(v11,v12 : (valid_finseq_list?)) :
       (accessed?(v11) AND accessed?(v12) AND v11 
       /= v12 =>
        FORALL(1 : pointer[P]) : value?(1) =>
        NOT list_member?(1,v11) OR NOT 
        list_member?(1,v12))
    \end{verbatim}
  \end{exampleblock}

\end{frame}

\subsection{Automatyczna weryfikacja warunkow spójności}

\begin{frame}[fragile]{Przykład}
  \begin{exampleblock}{TCC1 dla niepustosci listy, TCC2 dla zachowania typu w rekurencji, TCC3 dla warunku stopu rekurencji}
 
    \begin{verbatim}
      last_TCC1: OBLIGATION 
      FORALL (vl: (valid_finseq_list?)): NOT empty?(vl)
    \end{verbatim}

    \begin{verbatim}
      last_TCC2: OBLIGATION 
      FORALL (vl: (valid_finseq_list?)): 
        length(vl) > 1 IMPLIES valid_finseq_list?(tail(vl))
    \end{verbatim}

    \begin{verbatim}
      last_TCC3: OBLIGATION 
      FORALL (vl: (valid_finseq_list?)): 
        length(vl) > 1 IMPLIES length(tail(vl)) < length(vl)
    \end{verbatim}
  \end{exampleblock}
    
\end{frame}

\subsection{Półautomatyczne dowodzenie poprawności}

% \begin{frame}[fragile]{Przykładdd}

% \end{frame}


\subsection{Podsumowanie}
\begin{frame}
\begin{block}{Podsumowanie}
\begin{itemize}
\item Metoda skuteczna dla list i drzew.
\item Wymaga dużego nakładu pracy.
\item Obiecujące wyniki dla systemów krytycznych.
\end{itemize}
\end{block}
\end{frame}

\section{Zastosowanie metod formalnych: TLA+}
\begin{frame}{Czym jest TLA+?}
  \begin{itemize}
      \item TLA+ (Temporal Logic of Actions) - język specyfikacji formalnej opracowany przez Lesliego Lamporta.
      \item Służy do opisu systemów współbieżnych i rozproszonych.
      \item System definiowany jako zbiór zmiennych i akcji - przejść między stanami.
      \item Deklaratywny - skupia się na zachowaniu, nie na implementacji.
      \item Bazuje na logice temporalnej i matematyce zbiorów.
  \end{itemize}
\end{frame}

\begin{frame}{Narzędzie TLC i model checking}
  \begin{itemize}
      \item Weryfikacja poprawności specyfikacji odbywa się z użyciem narzędzia TLC.
      \item TLC używa techniki model checking:
      \begin{itemize}
          \item automatyczne przeszukiwanie przestrzeni stanów,
          \item sprawdzanie właściwości bezpieczeństwa i ciągłości.
      \end{itemize}
  \end{itemize}
\end{frame}

\begin{frame}[fragile]{Metoda 1: Modelowanie matematyczne (FIFO)}
  \textbf{Fragment specyfikacji bufora w TLA+:}
  \vspace{0.5em}

  \begin{verbatim}
VARIABLE queue

Init == queue = << >>

Enq(x) == queue' = Append(queue, x)

Deq == /\ queue # << >>
       /\ queue' = Tail(queue)
  \end{verbatim}

  \vspace{0.1em}
  \begin{itemize}
    \item Bufor jako lista (\texttt{queue}).
    \item Akcje opisują przejścia między stanami.
  \end{itemize}
\end{frame}

  
  
\begin{frame}[fragile]{Metoda 2: Inwariant bezpieczeństwa}
  \textbf{Fragment definicji inwariantu w TLA+:}
  \vspace{0.5em}

  \begin{verbatim}
TypeInvariant ==
  /\ queue \in Seq(Int)
  /\ Len(queue) >= 0

Inv == TypeInvariant
  \end{verbatim}

  \vspace{0.1em}
  \begin{itemize}
    \item Bufor jest sekwencją liczb całkowitych.
    \item Długość bufora nie może być ujemna.
    \item Inwarianty są sprawdzane automatycznie przez TLC.
    \item Chronią przed błędami, np. usunięciem z pustej kolejki.
  \end{itemize}
\end{frame}



\begin{frame}[fragile]{Metoda 3: Własności temporalne}
  \textbf{Fragment specyfikacji własności temporalnych w TLA+:}
  \vspace{0.5em}
  \begin{verbatim}
Safety == [](Len(queue) >= 0)

Liveness == <>(queue = << >>)
  \end{verbatim}

  \vspace{0.1em}
  \begin{itemize}
    \item \texttt{Safety} (bezpieczeństwo): długość bufora nigdy nie jest ujemna.
    \item \texttt{Liveness} (ciągłość): bufor w końcu się opróżni.
    \item Temporalne właściwości opisują zachowanie w czasie.
  \end{itemize}
\end{frame}

\begin{frame}{Zastosowania TLA+}
  \begin{itemize}
      \item Firmy: Amazon, Microsoft, Google.
      \item Projektowanie i weryfikacja systemów:
      \begin{itemize}
          \item rozproszonych,
          \item bazodanowych,
          \item komunikacyjnych,
          \item chmurowych.
      \end{itemize}
      \item Identyfikacja błędów przed implementacją - kluczowa dla systemów krytycznych.
  \end{itemize}
\end{frame}

\section{Zastosowanie metod formalnych w kolejnictwie}
\subsection{Dlaczego warto?}
\begin{frame}
\frametitle{Korzyści metod formalnych}
\begin{itemize}
\item Gwarancja bezpieczeństwa
\begin{itemize}
\item Wczesne wykrywanie błędów
\end{itemize}

\item Redukcja ryzyka
\begin{itemize}
\item Minimalizacja ludzkich pomyłek
\end{itemize}

\item Spełnienie standardów
\begin{itemize}
\item Normy CENELEC EN 50128 (SIL 3/4)
\end{itemize}

\item Optymalizacja kosztów
\begin{itemize}
\item Tańsze poprawianie na etapie projektowania
\end{itemize}

\item Zarządzanie złożonością
\begin{itemize}
\item Współbieżność, czas rzeczywisty, interakcje
\end{itemize}
\end{itemize}
\end{frame}

\subsection{Metoda B}
\begin{frame}
\frametitle{Metody formalne w sygnalizacji}
\begin{block}{Główne podejścia}
\begin{itemize}
\item Metoda B
\item Model Checking
\item Sieci Petriego
\item Dowodzenie Twierdzeń
\end{itemize}
\end{block}
\end{frame}

\begin{frame}
\frametitle{Metoda B}
\begin{itemize}
\item Specyfikacja (maszyny abstrakcyjne)
\item Stopniowa refinacja
\item Generacja kodu z weryfikacją
\end{itemize}
\end{frame}

\begin{frame}
\frametitle{Maszyny abstrakcyjne}

\begin{itemize}
    \item zmienne
    \item niezmienniki (inwarianty)
    \item operacje
\end{itemize}
\end{frame}

\begin{frame}{Specyfikacja maszyny abstrakcyjnej dla silni (Cz. 1/3)}
\begin{tcolorbox}[colback=white, colframe=blue!30!black, title=Machine FACTORIAL\_MAC]
\textbf{Sets:} \\
$\textit{factorial},\ m$

\textbf{Constants:} \\
$\textit{factorial}$

\textbf{Properties:}
\[
\begin{aligned}
& m \in \mathbb{N} \land \\
& \textit{factorial} \in \mathbb{N} \leftrightarrow \mathbb{N} \land \\
& f \in \mathbb{N} \leftrightarrow \mathbb{N} \land \\
& 0 \mapsto 1 \in f \land \\
\end{aligned}
\]
\end{tcolorbox}
\end{frame}

\begin{frame}{Specyfikacja maszyny abstrakcyjnej dla silni (Cz. 2/3)}
\begin{tcolorbox}[colback=white, colframe=blue!30!black, title=Machine FACTORIAL\_MAC]
\textbf{Properties (cont.):}
\[
\forall f.\ \begin{cases}
\forall (n, fn).\ (n \mapsto fn \in f \Rightarrow n+1 \mapsto (n+1) \times fn \in f) \\
\Rightarrow \\
\textit{factorial} \subseteq f
\end{cases}
\]

\textbf{Variables:}
\begin{itemize}
    \item $\textit{result}$
\end{itemize}

\textbf{Invariant:}
\[
\textit{result} \in \mathbb{N}
\]
\end{tcolorbox}
\end{frame}

\begin{frame}{Specyfikacja maszyny abstrakcyjnej dla silni (Cz. 3/3)}
\begin{tcolorbox}[colback=white, colframe=blue!30!black, title=Machine FACTORIAL\_MAC]
\textbf{Assertions:}
\[
\begin{aligned}
& \textit{factorial} \in \mathbb{N} \rightarrow \mathbb{N}; \\
& \textit{factorial}(0) = 1; \\
& \forall n.\,(n \in \mathbb{N} \Rightarrow \textit{factorial}(n+1) = (n+1) \times \textit{factorial}(n))
\end{aligned}
\]

\textbf{Initialisation:}
\[
\textit{result} := \mathbb{N}
\]

\textbf{Operations:}
\[
\textit{computation} = \text{\textbf{begin}}\ \textit{result} := \textit{factorial}(m)\ \text{\textbf{end}}
\]
\end{tcolorbox}
\end{frame}

\begin{frame}{Zastosowania i problemy}
\begin{exampleblock}{Zastosowania}
\begin{itemize}
\item Metro paryskie
\item Metro kopenhaskie
\item Sygnalizacja kolejowa
\end{itemize}
\end{exampleblock}

\begin{alertblock}{Wyzwania}
\begin{itemize}
\item Wymaga ekspertyzy
\item Czasochłonne
\end{itemize}
\end{alertblock}
\end{frame}

\subsection{Model Checking}
\begin{frame}
\frametitle{Model Checking}
\begin{itemize}
\item Automatyczna weryfikacja modelu
\item Graf stanów i własności systemu
\item Narzędzia: NuSMV, UPPAAL
\end{itemize}

\begin{exampleblock}{Przykładowe zastosowanie}
\begin{itemize}
\item System ERTMS
\item Weryfikacja bezpiecznych odległości
\end{itemize}
\end{exampleblock}

\begin{alertblock}{Ograniczenia}
Problem eksplozji stanów
\end{alertblock}
\end{frame}

\subsection{Porównanie metod}
\begin{frame}
\frametitle{Porównanie metod}
\scriptsize
\begin{tabularx}{\textwidth}{|l|X|X|}
\hline
\textbf{Metoda} & \textbf{Zalety} & \textbf{Wady} \\
\hline
Metoda B &
Pełna weryfikacja, generacja kodu &
Wymaga ekspertyzy, czasochłonne \\
\hline
Model Checking &
Automatyzacja, szybkość &
Problem eksplozji stanów \\
\hline
Sieci Petriego &
Wizualizacja współbieżności &
Słaba skalowalność \\
\hline
Dowodzenie &
Precyzja dla złożonych systemów &
Czasochłonne, manualne \\
\hline
\end{tabularx}
\end{frame}

\subsection{Podsumowanie}
\begin{frame}
\frametitle{Podsumowanie}
\begin{itemize}
\item Metody formalne - niezbędne narzędzie w kolejnictwie
\item Kluczowe dla spełnienia standardów bezpieczeństwa
\item Wybór metody zależy od:
\begin{itemize}
\item Złożoności systemu
\item Dostępnych zasobów
\item Wymagań standardów
\end{itemize}
\end{itemize}

\begin{block}{Przyszłe kierunki}
Integracja metod (np. Model Checking + Sieci Petriego)
\end{block}
\end{frame}

\section{Metody formalne w systemach wbudowanych}
\begin{frame}{Metody formalne w systemach wbudowanych}
  \begin{itemize}
    \item Metody formalne są kluczowe dla niezawodności i bezpieczeństwa systemów wbudowanych.
    \item Systemy wbudowane działają w środowiskach krytycznych, takich jak motoryzacja, medycyna czy lotnictwo.
    \item Wykorzystanie metod formalnych umożliwia:
      \begin{itemize}
        \item Precyzyjne modelowanie zachowań systemów.
        \item Weryfikację zgodności z rygorystycznymi normami.
        \item Walidację poprawności działania w różnych scenariuszach.
      \end{itemize}
    \item Zapobiega awariom, które mogą prowadzić do poważnych konsekwencji.
  \end{itemize}
\end{frame}

\begin{frame}{Metoda RT-EFSM – Wprowadzenie}
  \begin{block}{Czym jest RT-EFSM?}
    Rozszerzona Maszyna Stanów Skończonych Czasu Rzeczywistego (RT-EFSM) to formalny model stosowany do testowania oprogramowania wbudowanego czasu rzeczywistego,
    łączący elementy tradycyjnych FSM i EFSM z aspektami czasowymi.
  \end{block}
  \begin{block}{Kluczowe cechy modelu RT-EFSM}
    \begin{itemize}
      \item \textbf{Stany (\(S^*\))}
      \item \textbf{Zdarzenia wejściowe i wyjściowe (\(I, O\))}
      \item \textbf{Warunki strażnicze (\(C\))}
      \item \textbf{Zmienne środowiskowe (\(V\))}
      \item \textbf{Globalny zegar (\(L\))}
    \end{itemize}
  \end{block}
\end{frame}

\begin{frame}{Zastosowanie RT-EFSM w Testowaniu Systemów Wbudowanych}
  \begin{block}{Praktyczne zastosowania}
    Metoda RT-EFSM została skutecznie zastosowana w projektach takich jak testowanie systemu nawigacji bezwładnościowo-GPS dla lotnictwa,
    poprawiając jakość i niezawodność oprogramowania.
  \end{block}

  \begin{block}{Proces testowania RT-EFSM}
    \begin{itemize}
      \item Tworzenie klas równoważności przejść z ograniczeniami czasowymi.
      \item Generowanie drzew scenariuszy testowych i sekwencji testowych.
      \item Definiowanie przypadków testowych według kryteriów pokrycia.
    \end{itemize}
  \end{block}
\end{frame}

\begin{frame}{Sieci Petriego – Podstawy i Zalety}
  \begin{block}{Czym są Sieci Petriego?}
    Sieci Petriego (Petri Nets) to formalny model matematyczny używany do modelowania i weryfikacji systemów współbieżnych oraz czasowo-zależnych. Składają się z:
  \begin{itemize}
    \item \textbf{Miejsc (Places)} – reprezentujących stany lub zasoby.
    \item \textbf{Tranzycji (Transitions)} – przedstawiających zdarzenia lub akcje.
    \item \textbf{Łuków (Arcs)} – określających relacje pomiędzy miejscami i tranzycjami.
    \item \textbf{Tokenów} – wskazujących aktualny stan systemu.
  \end{itemize}
  \end{block}
\end{frame}

\begin{frame}{Zastosowania Sieci Petriego w systemach wbudowanych}
  \begin{block}{Przykłady zastosowań}
    \begin{itemize}
      \item \textbf{Kontrolery systemów krytycznych} – np. kontroler świateł drogowych, który wymaga zapewnienia, że zielone światła nie świecą się jednocześnie w przecinających się kierunkach.
      \item \textbf{Weryfikacja bezpieczeństwa} – narzędzia jak CPN Tools pozwalają sprawdzić zgodność systemu z zadanymi właściwościami bezpieczeństwa.
      \item \textbf{Wczesna weryfikacja} – możliwość formalnego potwierdzenia poprawności na wczesnych etapach projektowania systemu, co znacząco redukuje koszty.
    \end{itemize}
  \end{block}
\end{frame}

\begin{frame}{Metoda B – Wprowadzenie}
  \begin{block}{Czym jest Metoda B?}
    Metoda B to formalna metoda służąca do specyfikacji, projektowania i kodowania systemów oprogramowania. Bazuje na:
    \begin{itemize}
      \item Matematycznej teorii zbiorów Zermelo-Fraenkla.
      \item Logice predykatów pierwszego rzędu.
    \end{itemize}
  \end{block}
  
  \begin{block}{Kluczowe elementy metody B}
  \begin{itemize}
    \item \textbf{Specyfikacja i projektowanie}
    \item \textbf{Uściślenie (Refinement)}
    \item \textbf{Dowody matematyczne}
    \item \textbf{Język B}
  \end{itemize}
  \end{block}
\end{frame}

\begin{frame}{Zastosowania metody B w systemach wbudowanych}
  \begin{block}{Główne zastosowania}
  \begin{itemize}
    \item \textbf{Bezpieczne systemy wbudowane}
    \item \textbf{Jądro separacyjne}
    \item \textbf{Dowody poprawności}
    \item \textbf{Prototypowanie}
  \end{itemize}
  \end{block}
\end{frame}

\begin{frame}{Bibliografia}
  \scriptsize
  \begin{thebibliography}{9}
    \bibitem{lamport2002specifying}
    Leslie Lamport, \emph{Specifying Systems: The TLA+ Language and Tools for Hardware and Software Engineers}, Addison-Wesley, 2002.
  
    \bibitem{newcombe2015amazon}
    Chris Newcombe et al., \emph{How Amazon Web Services Uses Formal Methods}, Communications of the ACM, 2015.
  
    \bibitem{konnov2019tla}
    Igor Konnov, Jure Kukovec, Thanh-Hai Tran, \emph{TLA+ Model Checking Made Symbolic}, CAV 2019.
    
    \bibitem{newcombe2015formal}
    Chris Newcombe, Tim Rath, Fan Zhang, Bogdan Munteanu, Marc Brooker, Michael Deardeuff, \emph{How Amazon Web Services Uses Formal Methods}, Communications of the ACM, Vol. 58, No. 4, pp. 66--73, 2015.

    \bibitem{wayne2018practical}
    Hillel Wayne, \emph{Practical TLA+: Planning Driven Development}, Lospinato Books, 2018.
    
    \bibitem{paper}
    S. Poreda,
    \textit{Wykorzystanie metod formalnych do specyfikacji struktur wskaźnikowych},
    Uniwersytet Warszawski, 2023.
  \end{thebibliography}
\end{frame}

\begin{frame}{Bibliografia (cd.)}
  \scriptsize
  \begin{thebibliography}{9}
    \bibitem{spin_needham_schroeder}
    Sławomir Lasota,
    \textit{Weryfikacja protokołu Needhama-Schroedera przy użyciu narzędzi SPIN i UPPAAL},
    Wydział Matematyki, Informatyki i Mechaniki, Uniwersytet Warszawski,
    \url{https://www.mimuw.edu.pl/~sl/teaching/03_04/WPKWK/PREZENTACJE-SPIN_UPPAAL/NS/}.

    \bibitem{wojnicki2019weryfikacja}
    Igor Wojnicki,
    \textit{Weryfikacja własności systemów współbieżnych z użyciem metod formalnych},
    Praca doktorska, Akademia Górniczo-Hutnicza im. Stanisława Staszica w Krakowie, 2019,
    \url{https://winntbg.bg.agh.edu.pl/rozprawy2/10463/full10463.pdf}.

    \bibitem{b-original}
    J.R Abrial. \emph{Modeling in Event-B}, Cambridge University Press, 2013.

    \bibitem{b-other}
    Abrial, J.R., Lee, M.K.O., Neilson, D.S., Scharbach, P.N., Sørensen, I.H. \emph{The B-method}. VDM '91 Formal Software Development Methods, 1991. \url{https://doi.org/10.1007/BFb0020001}

    \bibitem{b-tutorial}
    Dominique Cansell, Dominique Méry. \emph{Tutorial on the event-based B method}. IFIP FORTE 2006, Paris, 2006.

    \bibitem{b-cenelec-railway}
    A. Abdullah, \emph{CENELEC EN 50128: Railway Applications - Communication, signaling and processing systems, Software for railway control and protection systems}, Frankfurt University of Applied Sciences, 2020.
  \end{thebibliography}
\end{frame}


\end{document}